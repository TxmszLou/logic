% Created 2016-04-04 Mon 22:00
\documentclass[11pt]{article}
\usepackage[utf8]{inputenc}
\usepackage[T1]{fontenc}
\usepackage{fixltx2e}
\usepackage{graphicx}
\usepackage{grffile}
\usepackage{longtable}
\usepackage{wrapfig}
\usepackage{rotating}
\usepackage[normalem]{ulem}
\usepackage{amsmath}
\usepackage{textcomp}
\usepackage{amssymb}
\usepackage{capt-of}
\usepackage{hyperref}
\usepackage{bussproofs} \usepackage[makeroom]{cancel}
\author{Sixuan Thomas Lou}
\date{\today}
\title{Notes on Mathematical Logic By Ian Chiswell and Wilfrid Hodges}
\hypersetup{
 pdfauthor={Sixuan Thomas Lou},
 pdftitle={Notes on Mathematical Logic By Ian Chiswell and Wilfrid Hodges},
 pdfkeywords={},
 pdfsubject={},
 pdfcreator={Emacs 24.5.1 (Org mode 8.3.4)}, 
 pdflang={English}}
\begin{document}

\maketitle
\tableofcontents

\newpage

\section{Basic Notions}
\label{sec:orgheadline25}
\subsection{Statement:}
\label{sec:orgheadline1}

Later formalized with first-order language

\subsection{set of statements}
\label{sec:orgheadline2}


\subsection{Sequent}
\label{sec:orgheadline4}

\[\Gamma \vdash \phi\]
  where \(\Gamma\) is a set of statements (assumptions)
        \(\phi\)   is a statement (conclusion)


The sequent is correct if and only if there is a proof of \(\phi\)
with no undischarged assumptions

\subsubsection{Sequent rules}
\label{sec:orgheadline3}

\begin{itemize}
\item Axiom:         If \(\phi\) \(\in\) \(\Gamma\), then \(\Gamma\) \(\vdash\) \(\phi\) is correct
\item Transitivity : If \(\Delta\) \(\vdash\) \(\phi\) and for all statements \(\delta\) in \(\Delta\),
the sequent \(\Gamma\) \(\vdash\) \(\delta\) is correct,
then \(\Gamma\) \(\vdash\) \(\phi\) is correct
\end{itemize}


\subsection{Natural Deduction}
\label{sec:orgheadline6}

Natural Deduction rules tells how to build a \uline{proof} for sequents,
especially called \uline{derivations}.

Derivations + rules of using them gives \uline{natural deduction calculus}

\subsubsection{Natural Deduction Rules}
\label{sec:orgheadline5}

\begin{itemize}
\item Axiom:
\end{itemize}
Let \(\phi\) be a statement, then
      \[ \phi \]
is a derivation whose conclusion is \(\phi\) , with one undischarged assumption, namely \(\phi\)


\subsection{Simple logical connectives with introduction / elimination rules}
\label{sec:orgheadline23}

\subsubsection{And (\(\wedge\))}
\label{sec:orgheadline9}

\begin{enumerate}
\item \textsc{Natural Deduction Rules}
\label{sec:orgheadline7}

\begin{itemize}
\item If \alwaysNoLine \AxiomC{$D$} \UnaryInfC{$\phi$} \DisplayProof and
\alwaysNoLine \AxiomC{$D'$} \UnaryInfC{$\psi$} \DisplayProof are derivations of \(\phi\) and \(\psi\) respectively,
then
\end{itemize}
\begin{prooftree}
  \noLine
  \AxiomC{$D$} \UnaryInfC{$\phi$}
  \noLine
  \AxiomC{$D'$} \UnaryInfC{$\psi$}
  \RightLabel{$(\wedge I)$}
  \BinaryInfC{$\phi \wedge \psi$}
\end{prooftree}
is a derivation of (\(\phi\) \(\wedge\) \(\psi\)).

\begin{itemize}
\item If  \noLine \AxiomC{$D$} \UnaryInfC{$\phi \wedge \psi$} \DisplayProof is a derivation of \((\phi \wedge \psi)\), then
\end{itemize}
\begin{prooftree}
  \noLine
  \AxiomC{$D$} \UnaryInfC{$\phi \wedge \psi$}
  \RightLabel{$(\wedge E)$}
  \UnaryInfC{$\phi$}
\end{prooftree}

\begin{prooftree}
  \noLine
  \AxiomC{$D$} \UnaryInfC{$\phi \wedge \psi$}
  \RightLabel{$(\wedge E)$}
  \UnaryInfC{$\psi$}
\end{prooftree}

are derivations of \(\phi\) and \(\psi\) respectively.


\item \textsc{Sequent Rules}
\label{sec:orgheadline8}

\begin{itemize}
\item If \(\Gamma \vdash \phi\) and \(\Delta \vdash \psi\) are correct sequents, then \(\Gamma \cup \Delta \vdash (\phi \wedge \psi)\) is a correct sequent.
\item If \(\Gamma \vdash (\phi \wedge \psi)\) is a correct sequent, then \(\Gamma \vdash \phi\) and \(\Gamma \vdash \psi\) are correct sequents.
\end{itemize}
\end{enumerate}

\subsubsection{Implication (\(\rightarrow\))}
\label{sec:orgheadline12}
\begin{enumerate}
\item \textsc{Natural Deduction Rules}
\label{sec:orgheadline10}

\begin{itemize}
\item If \noLine \AxiomC{$D$} \UnaryInfC{$\phi$} \DisplayProof is a derivation of \(\psi\) , \(\phi\) is a statement,
\end{itemize}
then 
\begin{prooftree}
  \noLine
  \AxiomC{$\cancel{\phi}$} \UnaryInfC{$D$} \noLine \UnaryInfC{$\psi$}
  \RightLabel{$(\rightarrow I)$}
  \UnaryInfC{$\phi \rightarrow \psi$}
\end{prooftree}
is a derivation of (\(\phi\) \(\rightarrow\) \(\psi\)).

\begin{itemize}
\item If \noLine \AxiomC{$D$} \UnaryInfC{$\phi$} \DisplayProof and \noLine \AxiomC{$D'$} \UnaryInfC{$\phi \rightarrow \psi$} \DisplayProof are derivations of \(\phi\) and \(\phi \rightarrow \psi\) respectively,
\end{itemize}
then
\begin{prooftree}
  \noLine
  \AxiomC{$D$} \UnaryInfC{$\phi$}
  \noLine
  \AxiomC{$D'$} \UnaryInfC{$\phi \rightarrow \psi$}
  \RightLabel{$(\rightarrow E)$}
  \BinaryInfC{$\psi$}
\end{prooftree}

is a derivation of \(\psi\).


\item \textsc{Sequent Rules}
\label{sec:orgheadline11}

\begin{itemize}
\item If \(\Gamma \cup \{\phi\} \vdash \psi\) is a correct sequent, then \(\Gamma \vdash (\phi \rightarrow \psi)\) is also correct.
\item If \(\Gamma \vdash (\phi \vdash \psi)\) and \(\Delta \vdash \phi\) are correct sequents, then so is \(\Gamma \cup \Delta \vdash \psi\).
\end{itemize}
\end{enumerate}


\subsubsection{Equivalence (\(\leftrightarrow\))}
\label{sec:orgheadline15}
\begin{enumerate}
\item \textsc{Natural Deduction Rules}
\label{sec:orgheadline13}

\begin{itemize}
\item If \noLine \AxiomC{$D$} \UnaryInfC{$\phi \rightarrow \psi$} \DisplayProof and \noLine \AxiomC{$D'$} \UnaryInfC{$\psi \rightarrow \phi$} \DisplayProof are derivations of \(\phi \rightarrow \psi\) and \(\psi \rightarrow \phi\) respectively,
\end{itemize}
then
\begin{prooftree}
  \noLine
  \AxiomC{$D$} \UnaryInfC{$\phi \rightarrow \psi$}
  \noLine
  \AxiomC{$D'$} \UnaryInfC{$\psi \rightarrow \phi$}
  \RightLabel{$(\leftrightarrow I)$}
  \BinaryInfC{$\phi \leftrightarrow \psi$}
\end{prooftree}
is a derivation of \((\phi \leftrightarrow \psi)\).

\begin{itemize}
\item If \noLine \AxiomC{$D$} \UnaryInfC{$\phi \leftrightarrow \psi$} \DisplayProof is a derivation of \((\phi \leftrightarrow \psi)\),
\end{itemize}
then
\begin{prooftree}
  \noLine
  \AxiomC{$D$} \UnaryInfC{$\phi \leftrightarrow \psi$}
  \RightLabel{$(\leftrightarrow E)$}
  \UnaryInfC{$\phi \rightarrow \psi$}
\end{prooftree}
\begin{prooftree}
  \noLine
  \AxiomC{$D$} \UnaryInfC{$\phi \leftrightarrow \psi$}
  \RightLabel{$(\leftrightarrow E)$}
  \UnaryInfC{$\psi \rightarrow \phi$}
\end{prooftree}
are derivations of \((\phi \rightarrow \psi)\) and \((\psi \rightarrow \phi)\) respectively.


\item \textsc{Sequent Rules}
\label{sec:orgheadline14}

\begin{itemize}
\item If \(\Gamma \vdash \phi \rightarrow \psi\) and \(\Delta \vdash \psi \rightarrow \phi\) are correct sequents, then so is \(\Gamma \cup \Delta \vdash \phi \leftrightarrow \psi\)
\item If \(\Gamma \vdash \phi \leftrightarrow \psi\) is a correct sequent, then so is \(\Gamma \vdash \phi \rightarrow \psi\) and \(\Gamma \vdash \psi \rightarrow \phi\)
\end{itemize}
\end{enumerate}

\subsubsection{Negation (\(\neg{}\))}
\label{sec:orgheadline19}

\begin{enumerate}
\item \textsc{Natural Deduction Rules}
\label{sec:orgheadline16}

\begin{itemize}
\item If \noLine \AxiomC{$D$} \UnaryInfC{$\bot$} \DisplayProof is a derivation if \(\bot\),
\end{itemize}
then
\begin{prooftree}
  \noLine
  \AxiomC{$\cancel{\phi}$}
  \UnaryInfC{$D$}
  \noLine
  \UnaryInfC{$\bot$}
  \RightLabel{$(\neg I)$}
  \UnaryInfC{$(\neg \phi)$}
\end{prooftree}
is a derivation of \((\neg \phi)\).

\begin{itemize}
\item If \noLine \AxiomC{$D$} \UnaryInfC{$\phi$} \DisplayProof and \noLine \AxiomC{$D'$} \UnaryInfC{$\neg \phi$} \DisplayProof are derivations of \(\phi\) and \((\neg \phi)\) respectively,
\end{itemize}
then
\begin{prooftree}
  \noLine
  \AxiomC{$D$} \UnaryInfC{$\phi$}
  \noLine
  \AxiomC{$D'$} \UnaryInfC{$\neg \phi$}
  \RightLabel{$(\neg E)$}
  \BinaryInfC{$\bot$}
\end{prooftree}
is a derivation of \(\bot\).

\begin{itemize}
\item If \noLine \AxiomC{$D$} \UnaryInfC{$\bot$} \DisplayProof is a derivation of \(\bot\),
\end{itemize}
then
\begin{prooftree}
  \noLine
  \AxiomC{$\cancel{(\neg \phi)}$}
  \UnaryInfC{$D$}
  \noLine
  \UnaryInfC{$\bot$}
  \RightLabel{(\textsc{raa})}
  \UnaryInfC{$\phi$}
\end{prooftree}
is a derivation of \(\phi\).

\item \textsc{Sequent Rules}
\label{sec:orgheadline17}

\begin{itemize}
\item If \(\Gamma \cup \{\phi\} \vdash \bot\) is a correct sequent, then \(\Gamma \vdash (\neg \phi)\) is also a correct sequent.
\item If \(\Gamma \vdash \phi\) and \(\Delta \vdash \neg \phi\) are correct sequents, then \(\Gamma \cup \Delta \vdash \bot\) is also correct.
\item If \(\Gamma \cup \{(\neg \phi)\} \vdash \bot\) is a correct sequent, then \(\Gamma \vdash \phi\) is also correct.
\end{itemize}



\item \textsc{Example.}
\label{sec:orgheadline18}
Find natural deduction proofs of the following sequent.
\[ \{(\neg (\phi \leftrightarrow \psi))\} \vdash (\neg \phi) \leftrightarrow \psi \]

\begin{prooftree}


  \AxiomC{$\cancel{\phi}$}
  \AxiomC{$\cancel{\neg \phi}$}
  \BinaryInfC{$\bot$}

  \UnaryInfC{$\psi$}
  \UnaryInfC{$\phi \rightarrow \psi$}

  \AxiomC{$\cancel{\psi}$}
  \AxiomC{$\cancel{\neg \psi}$}
  \BinaryInfC{$\bot$}

  \UnaryInfC{$\phi$}
  \UnaryInfC{$\psi \rightarrow \phi$}

  \BinaryInfC{$(\phi \rightarrow \psi) \wedge (\psi \rightarrow \phi)$}
  \UnaryInfC{$\phi \leftrightarrow \psi$}

  \AxiomC{$\neg (\phi \leftrightarrow \psi)$}

  \BinaryInfC{$\bot$}
  \UnaryInfC{$\psi$}
  \UnaryInfC{$(\neg \phi) \rightarrow \psi$}


  \AxiomC{$\cancel{\psi}$}
  \UnaryInfC{$\phi \rightarrow \psi$}

  \AxiomC{$\cancel{\phi}$}
  \UnaryInfC{$\psi \rightarrow \phi$}

  \BinaryInfC{$(\phi \rightarrow \psi) \wedge (\psi \rightarrow \phi)$}
  \UnaryInfC{$\phi \leftrightarrow \psi$}

  \AxiomC{$\neg (\phi \leftrightarrow \psi)$}

  \BinaryInfC{$\bot$}
  \UnaryInfC{$\neg \phi$}
  \UnaryInfC{$\psi \rightarrow (\neg \phi)$}

\BinaryInfC{$((\neg \phi) \rightarrow \psi) \wedge (\psi \rightarrow (\neg \phi))$}
\UnaryInfC{$(\neg \phi) \leftrightarrow \psi$}
\end{prooftree}
\end{enumerate}

\subsubsection{Or (\(\vee\))}
\label{sec:orgheadline22}

\begin{enumerate}
\item \textsc{Natural Deduction Rules}
\label{sec:orgheadline20}

\begin{itemize}
\item If \noLine \AxiomC{$D$} \UnaryInfC{$\phi$} \DisplayProof is a derivation of \(\phi\) and \(\psi\) is a statement, then
\end{itemize}
\begin{prooftree}
  \noLine
  \AxiomC{$D$}
  \UnaryInfC{$\phi$}
  \RightLabel{$(\vee I)$}
  \UnaryInfC{$\phi \vee \psi$}
\end{prooftree}
\begin{itemize}
\item If \noLine \AxiomC{$D$} \UnaryInfC{$\psi$} \DisplayProof is a derivation of \(\psi\) and \(\phi\) is a statement, then
\end{itemize}
\begin{prooftree}
  \noLine
  \AxiomC{$D$}
  \UnaryInfC{$\psi$}
  \RightLabel{$(\vee I)$}
  \UnaryInfC{$\phi \vee \psi$}
\end{prooftree}

\begin{itemize}
\item If \noLine \AxiomC{$D$} \UnaryInfC{$(\phi \vee \psi)$} \DisplayProof ,  \noLine \AxiomC{$D'$} \UnaryInfC{$\chi$} \DisplayProof and \noLine \AxiomC{$D''$} \UnaryInfC{$\chi$} \DisplayProof are derivations of \((\phi \vee \psi)\) and \(\chi\) respectively, then,
\end{itemize}
\begin{prooftree}
  \noLine
  \AxiomC{$D$} \UnaryInfC{$(\phi \vee \psi)$}
  \noLine
  \AxiomC{$\cancel{\phi}$} \UnaryInfC{$D'$} \noLine \UnaryInfC{$\chi$}
  \noLine
  \AxiomC{$\cancel{\phi}$} \UnaryInfC{$D''$} \noLine \UnaryInfC{$\chi$}
  \RightLabel{$(\vee E)$}
  \TrinaryInfC{$\chi$}
\end{prooftree}

\item \textsc{Sequent Rules}
\label{sec:orgheadline21}

\begin{itemize}
\item If \(\Gamma \vdash \phi\) is a correct sequent, \(\psi\) is a statement, then \(\Gamma \vdash (\phi \vee \psi)\) is also a correct sequent.
\item If \(\Gamma \vdash \psi\) is a correct sequent, \(\psi\) is a statement, then \(\Gamma \vdash (\phi \vee \psi)\) is also a correct sequent.
\item If \(\Gamma \cup \{\phi\} \vdash \chi\), \(\Delta \cup \{\psi\} \vdash \chi\) are correct sequents, then \(\Gamma \cup \Delta \cup \{(\phi \vee \psi)\} \vdash \chi\) is correct.
\end{itemize}
\end{enumerate}




\subsection{First order language}
\label{sec:orgheadline24}
\end{document}